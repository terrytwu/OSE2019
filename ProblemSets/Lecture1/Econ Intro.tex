\documentclass{beamer}

\mode<presentation>
{
  \usetheme{CambridgeUS}      % or try Darmstadt, Madrid, Warsaw, ...
  \usecolortheme{default} % or try albatross, beaver, crane, ...
  \usefonttheme{default}  % or try serif, structurebold, ...
  \setbeamertemplate{navigation symbols}{}
  \setbeamertemplate{caption}[numbered]
} 

\usepackage[english]{babel}
\usepackage[utf8x]{inputenc}

\usepackage{algorithm} 
\usepackage{algorithmic}  
\usepackage[algo2e]{algorithm2e} 


\usecolortheme{seagull}


\usepackage{ragged2e}
\usepackage{tgpagella}

\let\olditem=\item% 
\renewcommand{\item}{\olditem \justifying}%

\setbeamersize{text margin left=10mm,text margin right=10mm} 
\title[Econ]{\bf  Economics: A Brief Introduction}

\author{\textbf{Terry Wu\vspace{-10pt}}}

\date{\today}



\begin{document}
\begin{frame}
  \titlepage
\end{frame}

\begin{frame}
Road Map
\vspace{10pt}
\begin{itemize}
	\item Definition (15 min)
	\item Field Intro: Academic (15 min)
	\item Concepts (15 min)
	\item Basic Strategy (10 min)
	\item Advanced Strategy (10 min)
	\item Career Path (15 min)
	\item Frontiers (15 min)
	\item My Experience (10 min)
	\item PhD Overview (10 min)
	\item Application Tips (5 min)
\end{itemize}
\end{frame}

\begin{frame}{\bf Definition 1}
\begin{definition}
Economics is the social science that studies the optimal allocation of limited resource.
\end{definition}
\begin{example}
The position of the President is limited (one), how to design the best voting mechanism to select the one who perfectly serves the country? (Mechanism Design, Political Economy)
\end{example}
\end{frame}

\begin{frame}{\bf Definition 2}
\begin{definition}
Economics is the social science that studies the production, distribution, and consumption of goods and services.
\end{definition}
\begin{example}
A company plans to open up a new takeout platform, how much should they charge for an order and when to distributed bonus to selected consumers? (Consumer Theory, Industrial Organization)
\end{example}
\end{frame}


\begin{frame}{\bf Industrial Organization}
\begin{definition}
A field of economics dealing with the strategic behavior of firms, regulatory policy, antitrust policy and market competition.
\end{definition}
\begin{enumerate}
	\item What will the market price change after two firms merge?
	\item How to estimate the private value of bidders in a second price auction?
	\item What is the effect of introducing a new product into an well-established market?
\end{enumerate}
\end{frame}

\begin{frame}{\bf Econometrics}
\begin{definition}
	A field of economics applying statistical methods to economic data in order to give empirical content to economic relationships. 
\end{definition}
\begin{enumerate}
		\item How to deal the estimation with limited data points?
	\item What is the unbiased estimator in a multivariate regression model?
	\item How to identify the parameters in a discrete choice Logit model?
\end{enumerate}
\end{frame}


\begin{frame}{\bf International Economics (Trade)}
\begin{definition}
A field of study which assesses the implications of international trade in goods and services and international investment. 
\end{definition}
\begin{enumerate}
	\item What is the welfare implication of the US-China trade war?
	\item What is the major impact of canceling NAFTA?
	\item Is specialization of labor applied in development countries?
\end{enumerate}
\end{frame}

\begin{frame}{\bf Behavioral Economics}
\begin{definition}
A field of economic analysis that applies psychological insights into human behavior to explain economic decision-making.
\end{definition}
\begin{enumerate}
	\item How do people invest under uncertainty?
	\item Will non-proportional thinking induce large volatility in stock market?
	\item Does good company culture imply high performance?
\end{enumerate}
\end{frame}

\begin{frame}{\bf Labor Economics}
\begin{definition}
	A field that studies suppliers of labour services (workers) and the demanders of labour services (employers), and attempts to understand the resulting pattern of wages, employment, and income. 
\end{definition}
\begin{enumerate}
	\item Does large job posting rate imply high/low employment rate?
	\item Does consumption inequality mirror income inequality?
	\item When should a firm fire old employees?
\end{enumerate}
\end{frame}

\begin{frame}{\bf Health Economics}
\begin{definition}
	A field that concerns with issues related to efficiency, effectiveness, value and behavior in the production and consumption of health and healthcare.
\end{definition}
\begin{enumerate}
	\item Does match between physicians and patients affect treatment results?
	\item Will a merger of a hospital and a medical device supplier induce higher profits?
	\item Does automatic enrollment of health insurance improve consumer's welfare?
\end{enumerate}
\end{frame}

\begin{frame}{\bf Financial Economics (Finance)}
\begin{definition}
	A field that analyzes the use and distribution of resources in markets in which decisions are made under uncertainty. 
\end{definition}
\begin{enumerate}
	\item What is the optimal asset pricing of options and futures?
	\item Is risk diversification always better?
	\item Does agent's effect matters when listing your house?
\end{enumerate}
\end{frame}
\begin{frame}{\bf Other Fields}
Development Economics, Economic History, Economic Theory, Public Finance, Political Economy, Macroeconomics, Mathematical Economics, Urban Economics, Agricultural and Natural Resource Economics
\end{frame}

\begin{frame}{\bf Concept: Opportunity Cost}
\begin{definition}
Opportunity cost is the highest benefit missed when an investor, individual or business chooses one alternative over another.
\end{definition}
\begin{example}
Suppose you can work in a bank with a hourly rate \$50, what is your opportunity costs of staying here for a 2.5-hour class?
\end{example}
\end{frame}

\begin{frame}{\bf Concept: Opportunity Cost}
Challenge: Taylor Swift's concert tonight is free! However, you get another option of going to Coldplay's concert which \$100. You are not willing to pay more than \$120 for Coldplay's concert. What is your opportunity cost of going to Swift's concert?~(0, 100, 120, 20?)
\end{frame}

\begin{frame}{\bf Concept: Risk Aversion}
\begin{definition}
Risk aversion is the behavior of humans (especially consumers and investors), who, when exposed to uncertainty, attempt to lower that uncertainty.
\end{definition}
\begin{example}
When facing investment choices, risk averse investors are reluctant to buy those highly risky stocks, even which has higher returns with some probability.
\end{example}
\end{frame}

\begin{frame}{\bf Concept: Risk Aversion}
What would you choose to invest your \$10?
~\\
~\\
~\\
Portfolio A: certainly get back \$15 in the next period.
~\\
~\\
Portfolio B: get back \$100 with probability 0.2 and loss \$6.25 with probability 0.8.
\end{frame}

\begin{frame}{\bf Concept: Externality}
\begin{definition}
	Externality is side effect or consequence of an industrial or commercial activity that affects other parties without this being reflected in the cost of the goods or services involved.
\end{definition}
\begin{example}
Your noisy neighbor plays loud music every night; A nearby factory produces pollutions; Passive smoking...
\end{example}
Economic Solution: Taxation and Distributions, economy achieves Pareto optimum under ``invisible hands".
\end{frame}

\begin{frame}{\bf Basic Strategy: Utility and Production}
Utility is a mathematical function that ranks alternatives according to their utility to an individual.
\begin{example}
\begin{align*}
U(c)=\sum_i c_i, U(c)=\sum_i \log(c_i), U(c)=(\sum_ic_i^{\alpha})^{1/\alpha}
\end{align*}

\end{example}
\end{frame}

\begin{frame}{\bf Basic Strategy: Utility and Production}
Production function relates quantities of physical output of a production process to quantities of physical inputs.
\begin{example}
	\begin{align*}
F(K,L)=\alpha K^{\beta}L^{1-\beta}, F(K,L)=\gamma K+(1-\gamma)L
	\end{align*}
\end{example}
\end{frame}


\begin{frame}{\bf Basic Strategy: Optimization}
Oftentimes, we do the following maximization problem
\begin{align*}
\pmb{\theta}^*=&\arg \max f(x,\pmb{\theta}) \\
s.t.~~&\pmb{\theta}\in \Theta.
\end{align*}
The common tool we use is the Lagrangian method i.e.
\begin{align*}
\mathcal{L}=f(x,\pmb{\theta})-\lambda\times [Constrain].
\end{align*}
Then we take the derivative i.e.
\begin{align*}
\frac{\partial \mathcal{L}}{\partial \pmb{\theta}}|_{\pmb{\theta}=\pmb{\theta}^*}=0,\text{ under some regularity conditions.}
\end{align*}
\end{frame}

\begin{frame}{\bf Basic Strategy: Optimization Example}
Suppose we are deciding to spend $m$ dollars on two consumption goods with utility function $U(c_1,c_2)=\log(c_1)+\log(c_2)$. Suppose the price of $c_i$ is $p_i$. So we are solving the following problem
\begin{align*}
\max_{c_1,c_2} U(c_1,c_2)=&\log(c_1)+\log(c_2)\\
s.t.~&p_1c_1+p_2c_2=m.
\end{align*}
The Lagrangian is
\begin{align*}
\mathcal{L}=\log(c_1)+\log(c_2)-\lambda\times (p_1c_1+p_2c_2-m).
\end{align*}
\end{frame}

\begin{frame}{\bf Basic Strategy: Optimization Example}
Take derivative w.r.t. $c_1, c_2$ and $\lambda$ and we obtain
\begin{align*}
&p_1c_1+p_2c_2=m\\
&p_1\lambda c_1=1\\
&p_2\lambda c_2=1.
\end{align*}
Three unknowns with three equations, so we could solve them. The optimal consumptions are
\begin{align*}
c_1^*=\frac{m}{2p_1}\text{ and }c_2^*=\frac{m}{2p_2}.
\end{align*}
What if $p_1=p_2$, do you still need to solve the optimization problem?
\end{frame}

\begin{frame}{\bf Advanced Strategy: Panel Data Modeling}
Economists like panel data: they have rich information. The data is described as $\{x_{it},y_{it}\}_{i\in I, t\in T}.$ We run an reduced form estimation with fixed effect
\begin{align*}
y_{it}=\alpha +\beta x_{it}+d_i+\delta_t+\varepsilon_{it}
\end{align*}
where $x_{it}$ is covariate, $d_i$ is individual fixed effect, $\delta_t$ is time fixed effect and $\varepsilon_{it}$ is an error term.
\end{frame}

\begin{frame}{\bf Advanced Strategy: Computational Economics}
A sample R script:
\begin{figure}[H]
	\centering
\includegraphics[scale=0.36]{sample.png}
\end{figure}
\end{frame}

\begin{frame}{\bf Advanced Strategy: RCT}
Suppose we are interested in whether rising teacher's fixed income improve their attendance. We run an RCT in a development country and measure the before- and after- attendance. We should argue the effects are isolated, by controlling other effects.
\end{frame}

\begin{frame}{\bf Advanced Strategy: Dynamic Programming}
A cake-eating example: 
\begin{align*}
V(W_0)=&\max_{c_0}u(c_0)+\beta V(W_1)\\
s.t. &W_1=W_0-c_0,
\end{align*}
where $W_1$ is the cake left in next period, $W_0$ is the cake in this period and $c_0$ is the cake consumed in this period. 
~\\
~\\
How do we solve this problem?
\begin{enumerate}
	\item Guess-and-Verify. We guess $V(W_0)=a+bW_0$ and verify whether this holds.
	\item Value iteration. We guess an initial value function and iterate it over finite times until convergence.
\end{enumerate}
\end{frame}

\begin{frame}{\bf Advanced Strategy: Machine Learning and Data Mining}
\begin{enumerate}
	\item Natural Language Processing (LDA Topic Modeling): Hotel Review Data
	\item Regression Methods: Ridge and LASSO 
	\item Causal Inference: RCT for Predication, Support Vector Machine, Random forest, etc
\end{enumerate}
\end{frame}

\begin{frame}{\bf Career Path: Academia}
Teach students the principles of economics, Conduct research within your field, Formulate new theories and explanations of how markets work (or don’t work!)
\begin{enumerate}
\item Professor
\item Research Analyst (Economist)
\end{enumerate}
\end{frame}

\begin{frame}{\bf Career Path: Industry}
Quite flexible:
~\\
\begin{enumerate}
	\item Investment banks
	\item Consultancies
\item Major financial institutions
\item 	Financial advisories
\item Credit Analyst
\item Statistician
\item ......
\end{enumerate}
\end{frame}

\begin{frame}{\bf Frontiers}
\begin{enumerate}
	\item Statistical Learning and Data Mining
	\item Big Data
	\item Hotspot
   	\item Bayesian Econometrics
   	\item Social and Economic Networks
\end{enumerate}
\end{frame}

\begin{frame}{\bf My Experience}
\begin{enumerate}
	\item Foundations are important: Microeconomics, Macroeconomics and Econometrics
	\item Intuitions dominate maths
	\item Programming is of importance
	\item Collaboration oftentimes is efficient
	\item Understand both theory and empirics
\end{enumerate}
\end{frame}

\begin{frame}{\bf PhD Overview}
\begin{enumerate}
	\item First Year: Core Courses
	\item End of the First Year: Written Comprehensive Exams
	\item Second Year: Field Course and Empirical Training
	\item End of the Second Year: Oral Comprehensive Exams and Second Year Paper
	\item Third Years and Above: Prospectus Seminars
	\item Fourth Yeas and Above: Job Market Paper
\end{enumerate}
\end{frame}

\begin{frame}{\bf Application Tips}
\begin{enumerate}
	\item Standardized exams (SAT/GRE, TOEFL) matter but not that matter
	\item You should be good at math and statistics
	\item Enumerate some topics that interest referees
	\item Narrow your fields down
	\item Research experience is crucial (for graduate schools)
	\item Letters of recommendations are the priority
\end{enumerate}
\end{frame}



\section{End}
\begin{frame}{\bf End}
\begin{center}
	\Large{Thank You!}
\end{center}
\end{frame}





\end{document}
